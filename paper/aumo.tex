% Options for packages loaded elsewhere
\PassOptionsToPackage{unicode}{hyperref}
\PassOptionsToPackage{hyphens}{url}
%
\documentclass[
]{article}
\usepackage{lmodern}
\usepackage{amssymb,amsmath}
\usepackage{ifxetex,ifluatex}
\ifnum 0\ifxetex 1\fi\ifluatex 1\fi=0 % if pdftex
  \usepackage[T1]{fontenc}
  \usepackage[utf8]{inputenc}
  \usepackage{textcomp} % provide euro and other symbols
\else % if luatex or xetex
  \usepackage{unicode-math}
  \defaultfontfeatures{Scale=MatchLowercase}
  \defaultfontfeatures[\rmfamily]{Ligatures=TeX,Scale=1}
\fi
% Use upquote if available, for straight quotes in verbatim environments
\IfFileExists{upquote.sty}{\usepackage{upquote}}{}
\IfFileExists{microtype.sty}{% use microtype if available
  \usepackage[]{microtype}
  \UseMicrotypeSet[protrusion]{basicmath} % disable protrusion for tt fonts
}{}
\makeatletter
\@ifundefined{KOMAClassName}{% if non-KOMA class
  \IfFileExists{parskip.sty}{%
    \usepackage{parskip}
  }{% else
    \setlength{\parindent}{0pt}
    \setlength{\parskip}{6pt plus 2pt minus 1pt}}
}{% if KOMA class
  \KOMAoptions{parskip=half}}
\makeatother
\usepackage{xcolor}
\IfFileExists{xurl.sty}{\usepackage{xurl}}{} % add URL line breaks if available
\IfFileExists{bookmark.sty}{\usepackage{bookmark}}{\usepackage{hyperref}}
\hypersetup{
  pdftitle={Aumo - дигиталните касови бележки от бъдещето},
  pdfauthor={Симо Александров; Любо Любчев},
  hidelinks,
  pdfcreator={LaTeX via pandoc}}
\urlstyle{same} % disable monospaced font for URLs
\setlength{\emergencystretch}{3em} % prevent overfull lines
\providecommand{\tightlist}{%
  \setlength{\itemsep}{0pt}\setlength{\parskip}{0pt}}
\setcounter{secnumdepth}{-\maxdimen} % remove section numbering

\title{Aumo - дигиталните касови бележки от бъдещето}
\author{Симо Александров \and Любо Любчев}
\date{}

\begin{document}
\maketitle
\begin{abstract}
Тонове касови бележки биват създадени и веднага изхвърелни, като за
изработката им се използва \texttt{BPA\ (Bisphenol\ A)}, химикал вреден
за човека. Заедно за изработката на тази хартия е също нужна дървесна
маса, което означава, че хиляди декари гори биват отсичани годишно.

\texttt{Aumo} е мобилно приложение, придружено с хардуерно устройство и
уеб съврър, което цели да премахне хартиените касови бележки, като ги
замести с дигитални. Касовите апарати на магазини и заведения ще бъдат
оборудвани с \texttt{Aumo}. Клиентите ще да получат техните дигитални
касови бележки при допира на тяхното мобилно устройство (през мобилното
ни приложение) с \texttt{Aumo} чрез NFC (Near-Field Communication)
технология.

За мотив да се използва дигиталната касова бележека пред хартиения
еквавилент, потребителите ще бъдат възнаграждавани с точки, всеки път
когато клиентът предпочете \texttt{Aumo} пред традиционнтата касова
бележка. Тези точки могат да бъдат използвани за бонуси под формата на
намаления или материални награди осигурени от търговския обект.

Проектът е с приложен характер, все още е в процес на разработка и е от
сферата по информатика и информационни технологии. Идеята е измислена от
Симо Александров, а е реализирана от двамата автори.

Tons of paper receipts are produced and then immediately thrown, for the
creation of which is used \texttt{BPA\ (Bisphenol\ A)}, a human toxic
chemical. Thousands of forest decares need to be cut down, as wood is
another main component, required for the creation of paper receipts.

\texttt{Aumo} is a mobile application, accompanied by a hardware device
and a web server, which aims at removing paper receipts by replacing
them with a digital equivalent. Receipt printers of shops and
restaurants will be equiped with \texttt{Aumo}. Clients will take their
digital receipts by approaching their phone (through our mobile
application) to \texttt{Aumo}, by establishing a connection via NFC
(Near-Field Communication) technology.

Incentive for using the digital receipt, as opposed to the paper
alternative, will be points which users receive when choosing
\texttt{Aumo} over the traditional receipt. Points can be exchanged for
bonuses, which can either be discounts or physical items, provided by
the shop or restaurant.

The project has applicational nature, it is still under development and
belongs to the IT field. The idea was conceived by Simo Aleksandrov and
was realised by both of the authors.
\end{abstract}

\setdefaultlanguage{russian}
\setmainfont[Ligatures=TeX]{Times New Roman}
\newfontfamily\cyrillicfont{Times New Roman}[Script=Cyrillic]



\end{document}
